\documentclass[11pt]{article}
%Gummi|061|=)
\usepackage{hyperref}
\usepackage[spanish]{babel}
\usepackage[utf8]{inputenc}
\usepackage{pdfpages}
\title{\textbf{Práctica sobre Arquitecturas: LogProcessor}}
\author{Israel Pavón Maculet - Sergio Arroutbi Braojos}
\selectlanguage{spanish}
\date{\today}
\usepackage[bottom=14em]{geometry}
\usepackage{amsmath}
\usepackage{mathtools}

\begin{document}

\hypersetup
{   
pdfborder={0 0 0}
}
   
\maketitle

\pagebreak

\tableofcontents

\pagebreak

\section{Introducción}
En esta práctica se pretende realizar el diseño arquitectónico de una solución de recolección de logs. En particular, se trata de diseñar un sistema que a partir de dos tipos de logs, logs de sistema y logs de aplicación, con sus respectivos formatos, distintos entre sí, se genere un único tipo de log que pueda ser analizado posteriormente.

A priori, se generarán dos tipos de logs por parte de las aplicaciones y sistemas ya existentes. Posteriormente, dichos logs se recibirán por un sistema dedicado que los tratará de procesar a un único formato intermedio standard. Finalmente, habrá una etapa de análisis que permitirá realizar ciertas acciones en función del contenido del log en formato standard.

Este documento no pretende describir la práctica, ya que su enunciado puede consultarse en el siguiente link:

\url{https://drive.google.com/folderview?id=0BwwK87eKaiN_dlgzZ1pEamRwM1k&usp=sharing_eid&tid=0B2GOgs_VnHZuflF3bXJnUmZkTHdxbFJPNGNmZjE3SHhYWDljUC10eC12WE1nMXp2LXIxdU0}

Se prentende más bien enunciar aquellas decisiones de diseño arquitectural llevadas a cabo, teniendo en cuenta el tipo de aplicación que se pide, el patrón o conjunto de patrones arquitectónicos que se pueden aplicar para obtener llegar a una solución y las iteraciones realizadas hasta afinar la arquitectura seleccionada.

\pagebreak

\section{Justificación sobre la elección de la arquitectura}



\pagebreak

\section{Documentación: Modelo de 4+1 vistas}

\pagebreak

\section{Implementación}

\end{document}
